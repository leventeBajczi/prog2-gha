Benutzerdokumentation

\subsection*{Build}

Um die Projekt zu bilden, muss man diese Repository klonieren (z.\+B. von \href{https://github.com/leventeBajczi/prog2-gha/}{\tt Git\+Hub} herunterladen), und eine einzige {\ttfamily make} Befehl ausgeben im Root des Projektes. Diese wird zwei Foldern herstellen, {\ttfamily obj} und {\ttfamily build}, welche jeweils die .o Objektdateien und die hergestellte und laufbare Programm enthalten.

\subsection*{Laufen lassen}

Diese Programm lässt sich mit Hilfe von C\+LI Switches laufen lassen. Diese sind\+:
\begin{DoxyItemize}
\item {\ttfamily -\/p}\+: Eine zu füllende Sprachendatei ausschreiben in den folgenden \mbox{\hyperlink{class_datei}{Datei}} (z.\+B. {\ttfamily ./prog2 -\/p language.\+lang} wird eine Sample im File {\ttfamily language.\+lang} ausschreiben). Falls diese C\+LI Switch benutzt wird, wird der Programm nach der Filegenerierung sich schließen.
\item {\ttfamily -\/l}\+: Eine gefüllte Sprachendatei einladen und verarbeiten. Die Forme dieser \mbox{\hyperlink{class_datei}{Datei}} ist\+: 
\begin{DoxyCode}
\{
"Move B to A":"mov",
"Add B to A":"add",
"Substract B from A":"sub",
"Swap the upper and lower 4 bits":"swp",
"Shift left, insert 0":"sl0",
"Shift right, insert 0":"sr0",
"Jump to subroutine":"jsr",
"Push value to stack":"push",
"Pop value from stack":"pop"
\}
\end{DoxyCode}

\item {\ttfamily -\/f}\+: Kann nur als die letzte C\+LI Switch benutzt werden. Wird die nächste Dateien als Subroutine einladen und verarbeiten (können später benutzt werden). Diese Dateien sehen so aus\+: 
\begin{DoxyCode}
Name der Subroutine
<instruction>
<instruction>
...
\end{DoxyCode}
 Die letzte zwei Switches sind nicht nötig, ohne sie kann die Programm auch laufen gelassen, dann werden die Defaultwerte benutzt.
\end{DoxyItemize}

\subsection*{Instruktionen}

Der \mbox{\hyperlink{class_virtual_machine}{Virtual\+Machine}}, der benutzt wird, ist eine 8-\/bit Gerät. Wegen Zeitzwang wurde nur einige Instruktionen implementiert (nämlich mov, add, sub, swp, sl0, sr0, jsr, push und pop), aber die Addition von anderen Instruktionen ist sehr einfach (Siehe die Programmiererdokumentation für weitere Informationen). Diese Instruktionen sind im Defaultsprachendatei\+: 
\begin{DoxyCode}
mov A B: kopiert den Wert von B in A
add A B: addiert den Wert von B zu A
sub A B: substrahiert den Wert von B aus A
swp A: Austauscht den untere und obene 4 Bits
sl0 A: Verschiebt den Wert von A nach links mit 1 Bit
sr0 A: Verschiebt den Wert von A nach rechts mit 1 Bit
jsr <label>: Lässt den Instruktionen definiert im Subroutine "label" laufen 
push B: Drückt den Wert von B zum Stack
pop A: Popt den Topwert aus dem Stack im A 
\end{DoxyCode}
 Die Werte können sein\+:
\begin{DoxyItemize}
\item A\+: Register (r1-\/r16), speicher (Im Form (rN) addressiert)
\item B\+: Register (r1-\/r16), speicher (Im Form (rN) addressiert) oder Literal (0x\+NN)
\end{DoxyItemize}

Spezielle Instruktionen\+:
\begin{DoxyItemize}
\item {\ttfamily quit}\+: Schließt den Programm
\item {\ttfamily (r1)} -\/ {\ttfamily (r16)}\+: Schreibt den Wert der Speicherbereich addressiert durch einen Register auf dem Bildschirm
\item {\ttfamily r1} -\/ {\ttfamily r16}\+: Schreibt den Wert der Register auf dem Bildschirm
\end{DoxyItemize}

\subsection*{Ein Beispiel}

Die exakte Operation des Programmes zu zeigen steht hier folgender Beispiel für den Benutz des Applikations\+:
\begin{DoxyEnumerate}
\item Die Sprachendatei\+:

Jede \mbox{\hyperlink{class_instruktion}{Instruktion}} wird mit einem anderen ersetzt. Die Sprachendatei steht hier\+: 
\begin{DoxyCode}
\{
    "Move B to A":"ertek",
    "Add B to A":"hozzaad",
    "Substract B from A":"kivon",
    "Swap the upper and lower 4 bits":"csere",
    "Shift left, insert 0":"balra",
    "Shift right, insert 0":"jobbra",
    "Jump to subroutine":"ugras",
    "Push value to stack":"mentes",
    "Pop value from stack":"betoltes"
\}
\end{DoxyCode}

\item Die Subroutinendatei 
\begin{DoxyCode}
sbr
betoltes r1
hozzaad r1 r1
kivon r1 0x01
csere r1
balra r1
jobbra r1
mentes r1
\end{DoxyCode}

\item Der Laufsparametern

{\ttfamily ./prog2 -\/l language.\+lang -\/f sbr}
\item Die eingetippten Zeilen\+: 
\begin{DoxyCode}
ertek r10 0x50
mentes r10
r10
ugras sbr
betoltes r10
r10
\end{DoxyCode}

\item Die Ausgang\+: 
\begin{DoxyCode}
[levente@archlinux test]$ ./prog2 -l language.lang -f sbr
ertek r10 0x50
mentes r10
r10
Wert von r10: 0x50
ugras sbr
betoltes r10
r10
Wert von r10: 0x79
quit
\end{DoxyCode}
 
\end{DoxyEnumerate}